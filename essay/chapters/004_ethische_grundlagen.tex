
\subsection{Datenschutz Grundverordnung}

Die \emph{Datenschutz Grundverordnung} (DSGVO) ist eine EU-Verordnung, die den Umgang mit Personendaten regelt.
Wie bereits in \cref{sec:02:lawethics} besprochen, ist es auch nicht nutzlos, die Gesetzeslage zu betrachten.
Dies gilt insbesondere, wenn die ethischen Hintergründe bekannt sind, aus denen die Gesetze entstanden sind.

In der DSGVO finden wir in Artikel 1:
\blockquote{
    (1) Diese Verordnung enthält Vorschriften zum Schutz natürlicher Personen bei der Verarbeitung personenbezogener Daten und zum freien Verkehr solcher Daten.

    (2) Diese Verordnung schützt die Grundrechte und Grundfreiheiten natürlicher Personen und insbesondere deren Recht auf Schutz personenbezogener Daten.

    (3) Der freie Verkehr personenbezogener Daten in der Union darf aus Gründen des Schutzes natürlicher Personen bei der Verarbeitung personenbezogener Daten weder eingeschränkt noch verboten werden.
}

Wir sehen hier, dass als Grundlage für die DSGVO gilt, dass jede Person ein \enquote{Recht auf Schutz personenbezogener Daten} besitzt (Nr. 2), 
natürliche Personen, in Hinblick auf die Verarbeitung und Verkehr der persönlichen Daten, schutzbedürftig ist (Nr. 3) und diese Rechte zu schützen sind (Nr. 1).

Wir betrachten Auszüge aus der DSGVO, die für dieses Fallbeispiel relevant sind:

Sollen die Daten einer Person verarbeitet werden, so muss die Art und der Umfang der Verarbeitung für diese Person nachvollziehbar und transparent beschrieben werden (Art. 5 Nr. 1a) DSGVO).

Darüber hinaus muss ein legitimer Zweck (oder mehrere), dem die Sammlung und Verarbeitung der Daten gewidmet ist, eindeutig und verständlich festgehalten werden. Eine diesem Zweck nicht entsprechende Verarbeitung der Daten ist unzulässig (Art. 5 Nr. 1b) DSGVO). Natürlich muss die Person der Verarbeitung der Daten zu diesem Zweck zustimmen (Art. 6 Nr. 1a) DSGVO).
Damit geht einher, dass die gesammelten Daten auf das Notwendige beschränkt sind.
Demnach dürfen keine Daten gesammelt werden, die für den festgelegten Zweck nicht vonnöten sind (Art. 5 Nr. 1c) DSGVO).

Auch wenn Daten bereits erhoben sind, sind sie nun nicht Eigentum derjenigen Instanz, die diese erhoben hat. Denn sollten bereits erhobene Daten nicht mehr zur Erfüllung des Zwecks notwendig sein, müssen diese im Allgemeinen gelöscht werden (Art. 5 Nr. 1e) DSGVO).
Dabei muss sichergestellt werden, dass die Daten vor unberechtigtem Zugriff und unzulässiger Verarbeitung geschützt sind (Art. 5 Nr. 1f) DSGVO).

Die Instanz, die die Daten erhebt und verarbeitet ist dafür verantwortlich, die angesprochenen Regelungen einzuhalten (Art. 5 Nr. 2 DSGVO).


\subsection{International Data Privacy Principles}

Die \emph{International Data Privacy Principles} \cite{zankl_international_2014} (IDPP) sind 13 ethische Prinzipien für den Umgang mit Daten.
Diese wurden von Wolfgang Zankl vom Institut für Zivilrecht in Wien erarbeitet und in Harvard University\footnote{\url{https://www.harvard.edu/}}, sowie in der Computer Ethics Society Hong Kong\footnote{\url{http://www.iethicssoc.org/}} diskutiert.
Sie beziehen die Datenschutzstandards verschiedener Länder in Amerika, Asien, Europa, sowie internationale Datenschutzstandards mit ein.

Auch hier betrachten wir nur die Prinzipien, die zu diesem Fallbeispiel passen.

\paragraph*{Prinzip 1}

Eine Firma, die persönliche Daten nutzt, soll die jeweiligen nationalen Regelungen einhalten, die Datenschutz als Gegenstand haben.

\paragraph*{Prinzip 2}

Eine Firma, die persönliche Daten nutzt, soll durch Einhalten aktueller Sicherheitsstandards, unbefugten Zugriff und Verarbeitung, sowie Löschung und Verlust zu verhindern.

\paragraph*{Prinzip 3}

Eine Firma, die persönliche Daten nutzt, soll eine für Kunden einfach verständliche Datenschutzrichtlinie aufstellen und den Kunden erlauben, die verantwortliche Person persönlich zu kontaktieren.
Darüber hinaus muss die Richtlinie beschreiben, welche Daten zu welchem Zweck erhoben werden [und] wie die Daten genutzt werden [\dots].

\paragraph*{Prinzip 5}

Eine Firma, die persönliche Daten nutzt, soll Daten nicht an unautorisierte Dritte weiter geben, es sei denn [\dots] der Kunde hat zugestimmt.

\paragraph*{Prinzip 6}

Eine Firma, die persönliche Daten nutzt, soll nur nötige Daten sammeln.

\paragraph*{Prinzip 7}

Eine Firma, die persönliche Daten nutzt, soll die gesammelten Daten auf faire Art und Weise benutzen [\dots] und nur zum Zweck, der für die Tätigkeit der Firma nötig ist.

\paragraph*{Prinzip 8}

Eine Firma, die persönliche Daten nutzt, soll keine Daten an Instanzen weiter geben, die internationale Datenschutzstandards nicht einhalten.

\paragraph*{Prinzip 10}

Eine Firma, die persönliche Daten nutzt, soll Daten nur so lang wie nötig aufbewahren.

\paragraph*{Prinzip 13}

Eine Firma, die persönliche Daten nutzt, soll in Abwesenheit eines Vertrags, der den Kunden dazu verpflichtet, Dienste oder Güter zu kaufen:

\begin{itemize}
    \item den Nutzer im Fall eines Datenlecks so schnell wie möglich informieren, sollten persönliche Daten enthalten sein.
    \item den Nutzer auf Nachfrage über jegliche persönliche Daten informieren, die die Firma über diesen Nutzer speichert, und solche Daten auf Nachfrage zu löschen, sollten diese veraltet sein, es sei denn die Gesetzeslage verbietet dies
    \item persönliche Daten nicht ohne ausdrückliche, explizite und individuelle Zustimmung verwenden oder weitergeben.
\end{itemize}

\subsection{Utilitarismus}

Zuletzt betrachten wir den \emph{Utilitarismus} als Grundlage.
Der Utilitarismus ist eine von Jeremy Bentham begründete teleologische Ethik. Das bedeutet, dass die Bewertung einer Handlung davon abhängt,
welche Konsequenzen die Tat mit sich bringt. \cite{noauthor_teleological_nodate}

Bentham beschreibt das Nützlichkeitsprinzip (orig. Principle of Utility) folgendermaßen als Einleitung seines Werks \emph{An Introduction to the Principles of Morals and Legislation}:
\blockquote[\cite{bentham_principle_1780}, zitiert nach \cite{bensch_philosophisches_1984}]{
Die Natur hat die Menschheit unter die Herrschaft zweier souveräner Gebieter – Leid und Freude – gestellt. Es ist an ihnen allein aufzuzeigen, was wir tun sollen, wie auch zu bestimmen, was wir tun werden. 
Sowohl der Maßstab für Richtig und Falsch als auch die Kette der Ursachen und Wirkungen sind an ihrem Thron festgemacht. Sie beherrschen uns in allem, was wir tun, was wir sagen, was wir denken.
}

Im Fall des Utilitarismus ist eine Handlung also \enquote{gut}, wenn sie Glück (beziehungsweise Freude) fördert und \enquote{schlecht}, wenn sie Unglück oder Leid fördert.
Dies mag dem Hedonismus ähnlich klingen, allerdings orientiert sich der Utilitarismus an dem größten Nutzen für die größte Anzahl von Menschen, anstatt nur auf sich selbst gerichtet zu sein. \cite{white_principle_2001}



