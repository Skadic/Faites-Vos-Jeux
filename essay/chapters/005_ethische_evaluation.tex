\section{Ethische Evaluation}

Wir möchten nun zunächst den Fokus auf einige Fragestellungen bezüglich dieses Fallbeispiels legen. Hierfür werden wir folgende Hypothese Annehmen:

\begin{center}
\parbox{0.8\textwidth}{
    Data Broker wird die Daten der Kunden benutzen um, ähnlich wie Hank, Einfluss auf das Verhalten zu nehmen und ihren Profit zu maximieren. 
}
\end{center}

Dies ist zwar Spekulation, aber angesichts der dargestellten Situation bei weitem nicht unwahrscheinlich.
Die Annahme dieser Hypothese dient dazu, die Beantwortung der folgenden Fragen etwas sinnvoller zu gestalten, da die Antworten ohnehin teilweise auf Spekulation basieren müssen.

\paragraph*{Sollte Walter AC-Games an Data Broker verkaufen?\cite{kees_faites_2017} Gibt es eine Alternative?}

Aus rein unternehmerischer Perspektive ist die Antwort: Ja, er sollte verkaufen.
Es ist zu diesem Zeitpunkt die zuverlässigste Möglichkeit, die es AC-Games erlaubt weiterzuexistieren.

Aus ethischer Perspektive ist dies etwas schwieriger zu beantworten.
Es ist offensichtlich, dass Data Broker von diesen Daten profitieren möchte.
Allerdings wissen wir nicht, was Data Broker \emph{genau} mit diesen Daten vorhat und wie vertretbar Data Brokers Vorhaben sind.

Der wichtigste gegen den Verkauf ist, dass Data Broker versucht, mithilfe der Daten, die Kunden negativ zu beeinflussen. 
Im Fallbeispiel hat Hank gezeigt, dass dies durchaus möglich ist.
Darüber hinaus konnte Hank allein bereits die Kunden beeinflussen.
Eine Firma, dessen Expertise im Handel und Profitieren von Daten liegt, wird wahrscheinlich viel größere Kapazitäten haben als Hank, die Nutzer stärker zu beeinflussen.
Hier wird vorsätzlich und dauerhaft zum Nachteil der Kunden gehandelt und zum alleinigen Vorteil von Data Broker. Aus diesem Grund kann dies utilitaristisch als unethisch angesehen werden.

Zusätzlich dazu würde dies gegen Prinzip 7 der ISPP verstoßen. Im Begriff einer \enquote{fairen Art und Weise} mag es zwar Grauzonen geben, allerdings ist dieses Verhalten meines ermessens eindeutig nicht so zu bezeichnen.  


\subparagraph*{Alternativen}

Eine Alternative könnte Crowdfunding sein. 
Auch wenn es mit dieser Lösung Schwierigkeiten geben könnte (siehe die nächste Frage),
falls die Produkte, die AC-Games anbietet wirklich Wert für die Nutzer hat, dann werden diese möglicherweise willens sein, dafür zu spenden.

Eine weitere Alternative ist etwas anderer Art.
Walter hat über die letzten \emph{zehn Jahre} gemerkt, dass es mit dem Spieleportal und damit der Firma im Allgemeinen bergab geht.
Die Finanzkrise lag außer der Kontrolle von AC-Games lag. Aber Walter hatte auch gemerkt, 
dass AC-Games' Nutzerzahlen unter Anderem auf Grund von App-Stores sanken. 
Diese erfüllen quasi den Zweck von solch einem Spieleportal auf möglicherweise bessere Art und Weise.
Smartphones kamen ebenfalls im Jahr 2007 auf den Markt.
Spätestens einige Jahre später war es ersichtlich, dass Smartphones eine sehr wichtige Plattform für Spiele bieten.
Der Fakt, dass Walter diese Entwicklung ein ganzes Jahrzehnt beobachtet hat, ohne zu versuchen diesen Markt zu erreichen, 
hat möglicherweise dazu beigetragen, dass das Spieleportal an Nutzern verlor.

Beispielsweise hätte AC-Games ein Publisher für Spiele in App-Stores werden können.
AC-Games hätte eine höhere Wahrscheinlichkeit auch heute noch eine Firma mit stabilem Nutzerstamm und Einkommen zu sein.
Diese Ratschläge sind zu diesem Zeitpunkt zwar zu spät, hätte Walter aber versucht zu innovieren, hätte dies die jetzige Situation verhindern können.

\paragraph*{Sollte AC-Games sein Spieleportal weiterbetreiben trotz finanzieller Schwierigkeiten?\cite{kees_faites_2017}}

Prinzipiell ist gegen dieses Vorhaben nicht viel einzuwenden.
Aus utilitaristischer Sicht ist größerer Nutzen sowohl für AC-Games als auch für die Kunden durch den Weiterbetrieb möglich, sollte dieser ein Erfolg sein.
Sollte der Weiterbetrieb scheitern, so ist dies schlecht für sowohl AC-Games als auch die Kunden.

Wie bereits bei der vorherigen Frage angedeutet, könnte Crowdfunding als Lösung einige Probleme mit sich bringen.
Die finanziellen Schwierigkeiten stammen nicht nur von der Finanzkrise, sondern auch durch Veränderungen am Markt (App-Stores etc.).
Dies bedeutet, dass das Spieleportal für viele ehemalige Kunden keinen Mehrwehrt mehr bietet.
Damit könnte das Crowdfunding des Projekts schwierig werden. 

Man könnte die Frage stellen ob es ethisch ist ein Produkt weiterbetreiben zu wollen, für das viele Kunden mangelnden Nutzen sehen, doch meines Ermessens ist das nur ein schwacher Einwand. 
Auch utilitaristisch ist dies unproblematisch. Kunden sind nicht gezwungen das Produkt zu benutzen.
Kunden, die tatsächlich Mehrwehrt in dem Produkt sehen, können es verwenden und sowohl die Kunden als auch AC-Games profitieren.
Falls das Produkt nicht erfolgreich ist, hat im schlimmsten Fall AC-Games selbst Verluste.

\paragraph*{Welche moralische Verantwortung hat Walter, seine Kunden vor Data Brokers Übernahme zu warnen?\cite*{kees_faites_2017}}

Wir haben gesehen, dass Data Broker beabsichtigt, die Übernahme möglichst vor den Nutzern geheimzuhalten.
Auch wird Data Broker die Nutzer negativ in ihrem Kaufverhalten beeinflussen.
Würde Walter die Kunden \emph{nicht} informieren, so werden sie diesem Einfluss unwissend ausgesetzt.
Anders würden die Kunden wissen, dass Data Broker die Firma übernimmt und könnten beispielsweise ihr Konto löschen.
Dies hätte für die Kunden einen höheren Nutzen.

Auf der anderen Seite steht nur der Profit von AC-Games und damit Data Broker.
Natürlich kann es sein, dass durch Walters Warnung tatsächlich viele Kunden das Spieleportal verlassen und AC-Games noch weniger Einnahmen hat.
Dies könnte dazu führen, dass AC-Games bankrott geht, trotz Übernahme durch Data Broker.

Andererseits kann es schwierig sein, die Übernahme dauerhaft geheim zu halten.
Wie in \cref{sec:02:lawethics} besprochen, sind \emph{Soft Costs} auch wichtig zu betrachten, wenn die Konsequenzen einer solchen Entscheidung zu evaluieren sind.
Zwar können auch durch Walters Warnung, wie gerade besprochen, Soft Costs entstehen.
Wenn allerdings versucht wird, die Übernahme geheim zu halten, diese aber später bekannt, dann entstehen noch ernstere Soft Costs, da AC-Games nun zusätzlich mit Recht als hinterlistig bezeichnet werden kann.

Es ist etwas schwieriger hier eine klare Antwort zu finden. 
Allgemein wäre es meines Ermessens utilitaristisch das Richtige, die Kunden zu informieren.
Die Übernahme geheimzuhalten garantiert nicht, dass die Übernahme auch tatsächlich geheim bleibt.
Falls dies später ans Licht kommt, so resultiert dies in ähnlichem Nutzen für die Kunden als wenn Walter sie gleich informiert hätte.
Für AC-Games und Data Broker sind die Nutzen allerdings schlechter.

Hier könnte potentiell Prinzip 13 der ISPP angewandt werden. Dieses Prinzip bezieht sich darauf, dass bei einem \emph{Datenleck} der Nutzer informiert werden soll.Allerdings ist die Situation hier ähnlich genug um derselben ethischen Überlegung zu entsprechen.
Auch hier gelangen die persönlichen Daten des Nutzers unvorhergesehen in die Hände einer anderen Instanz, die auch dubiose Ziele mit diesen Daten verfolgt.

Um den ISPP zu entsprechen sollte Walter ebenfalls die Kunden informieren.

\paragraph*{Welche Verantwortung hatte Kathleen, die Datensammlung zu beenden nachdem die Daten so lang nicht benutzt wurden?}
