\section*{Ethische Evaluation}

Wir möchten nun zunächst den Fokus auf einige Fragestellungen bezüglich dieses Fallbeispiels legen.

\paragraph*{Sollte Walter AC-Games an Data Broker verkaufen? Gibt es eine Alternative?}

Aus rein unternehmerischer Perspektive ist die Antwort: Ja, er sollte verkaufen.
Es ist zu diesem Zeitpunkt die zuverlässigste Möglichkeit, die es AC-Games erlaubt weiterzuexistieren.

Aus ethischer Perspektive ist dies etwas schwieriger zu beantworten.
Es ist offensichtlich, dass Data Broker von diesen Daten profitieren möchte.
Allerdings wissen wir nicht, was Data Broker \emph{genau} mit diesen Daten vorhat und wie vertretbar Data Brokers Vorhaben sind.

Hier kann potentiell Prinzip 8 ISPP in Betracht gezogen werden.
Da Data Broker von den Daten profitieren möchte, ist es möglich, dass Data Broker beispielsweise die Daten an andere Instanzen weiterverkauft 
oder versucht, mithilfe der Daten, die Kunden negativ zu beeinflussen. (Dies würde gegen Prinzip 7 verstoßen)

Im Fallbeispiel hat Hank gezeigt, dass Letzteres durchaus möglich ist.
Darüber hinaus konnte Hank allein bereits die Kunden beeinflussen.
Eine Firma, dessen Expertise im Handel und Profitieren von Daten liegt, wird höchstwahrscheinlich viel größere Kapazitäten haben als Hank, die Nutzer stärker zu beeinflussen.
Allerdings basiert dies auf Spekulation, die aber meines Ermessens naheliegend ist.

Eine Alternative könnte Crowdfunding sein. 
Auch wenn es mit dieser Lösung Schwierigkeiten geben könnte (siehe die nächste Frage),
falls die Produkte, die AC-Games anbietet wirklich Wert für die Nutzer hat, dann werden diese möglicherweise willens sein, dafür zu spenden.

Eine weitere Alternative ist etwas anderer Art.
Walter hat über die letzten \emph{zehn Jahre} gemerkt, dass es mit dem Spieleportal und damit der Firma im Allgemeinen bergab geht.
Die Finanzkrise lag außer der Kontrolle von AC-Games lag. Aber Walter hatte auch gemerkt, 
dass AC-Games' Nutzerzahlen unter Anderem auf Grund von App-Stores sanken. 
Diese erfüllen quasi den Zweck von solch einem Spieleportal auf möglicherweise bessere Art und Weise.
Smartphones kamen ebenfalls im Jahr 2007 auf den Markt.
Spätestens einige Jahre später war es ersichtlich, dass Smartphones eine sehr wichtige Plattform für Spiele bieten.
Der Fakt, dass Walter diese Entwicklung ein ganzes Jahrzehnt beobachtet hat, ohne zu versuchen diesen Markt zu erreichen, 
hat möglicherweise dazu beigetragen, dass das Spieleportal an Nutzern verlor.

Beispielsweise hätte AC-Games ein Publisher für Spiele in App-Stores werden können.
AC-Games hätte eine höhere Wahrscheinlichkeit auch heute noch eine Firma mit stabilem Nutzerstamm und Einkommen zu sein.
Diese Ratschläge sind zu diesem Zeitpunkt zwar zu spät, hätte Walter aber versucht zu innovieren, hätte dies die jetzige Situation verhindern können.

\paragraph*{Sollte AC-Games sein Spieleportal weiterbetreiben trotz finanzieller Schwierigkeiten?}
