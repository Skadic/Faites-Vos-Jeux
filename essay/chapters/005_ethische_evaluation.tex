\section{Ethische Evaluation}

Wir möchten nun zunächst den Fokus auf einige Fragestellungen bezüglich dieses Fallbeispiels legen. Hierfür werden wir folgende Hypothese Annehmen:

\begin{center}
\parbox{0.8\textwidth}{
    Data Broker wird die Daten der Kunden benutzen um, ähnlich wie Hank, Einfluss auf das Verhalten zu nehmen und ihren Profit zu maximieren. 
}
\end{center}

Dies ist zwar Spekulation, aber angesichts der dargestellten Situation bei weitem nicht unwahrscheinlich.
Die Annahme dieser Hypothese dient dazu, die Beantwortung der folgenden Fragen etwas sinnvoller zu gestalten, da die Antworten ohnehin teilweise auf Spekulation basieren müssen.

\paragraph*{Sollte Walter AC-Games an Data Broker verkaufen? \cite{kees_faites_2017} Gibt es eine Alternative?}

Aus rein unternehmerischer Perspektive ist die Antwort: Ja, er sollte verkaufen.
Es ist zu diesem Zeitpunkt die zuverlässigste Möglichkeit, die es AC-Games erlaubt weiterzuexistieren.

Aus ethischer Perspektive ist dies etwas schwieriger zu beantworten.
Es ist offensichtlich, dass Data Broker von diesen Daten profitieren möchte.
Allerdings wissen wir nicht, was Data Broker \emph{genau} mit diesen Daten vorhat und wie vertretbar Data Brokers Vorhaben sind.

Der wichtigste gegen den Verkauf ist, dass Data Broker versucht, mithilfe der Daten, die Kunden negativ zu beeinflussen. 
Im Fallbeispiel hat Hank gezeigt, dass dies durchaus möglich ist.
Darüber hinaus konnte Hank allein bereits die Kunden beeinflussen.
Eine Firma, dessen Expertise im Handel und Profitieren von Daten liegt, wird wahrscheinlich viel größere Kapazitäten haben als Hank, die Nutzer stärker zu beeinflussen.
Hier wird vorsätzlich und dauerhaft zum Nachteil der Kunden gehandelt und zum alleinigen Vorteil von Data Broker. Aus diesem Grund kann dies utilitaristisch als unethisch angesehen werden.

Zusätzlich dazu würde dies gegen Prinzip 7 der ISPP verstoßen. Im Begriff einer \enquote{fairen Art und Weise} mag es zwar Grauzonen geben, allerdings ist dieses Verhalten meines ermessens eindeutig nicht so zu bezeichnen.  


\subparagraph*{Alternativen}

Eine Alternative könnte Crowdfunding sein. 
Auch wenn es mit dieser Lösung Schwierigkeiten geben könnte (siehe die nächste Frage),
falls die Produkte, die AC-Games anbietet wirklich Wert für die Nutzer hat, dann werden diese möglicherweise willens sein, dafür zu spenden.

Eine weitere Alternative ist etwas anderer Art.
Walter hat über die letzten \emph{zehn Jahre} gemerkt, dass es mit dem Spieleportal und damit der Firma im Allgemeinen bergab geht.
Die Finanzkrise lag außer der Kontrolle von AC-Games lag. Aber Walter hatte auch gemerkt, 
dass AC-Games' Nutzerzahlen unter Anderem auf Grund von App-Stores sanken. 
Diese erfüllen quasi den Zweck von solch einem Spieleportal auf möglicherweise bessere Art und Weise.
Smartphones kamen ebenfalls im Jahr 2007 auf den Markt.
Spätestens einige Jahre später war es ersichtlich, dass Smartphones eine sehr wichtige Plattform für Spiele bieten.
Der Fakt, dass Walter diese Entwicklung ein ganzes Jahrzehnt beobachtet hat, ohne zu versuchen diesen Markt zu erreichen, 
hat möglicherweise dazu beigetragen, dass das Spieleportal an Nutzern verlor.

Beispielsweise hätte AC-Games ein Publisher für Spiele in App-Stores werden können.
AC-Games hätte eine höhere Wahrscheinlichkeit auch heute noch eine Firma mit stabilem Nutzerstamm und Einkommen zu sein.
Diese Ratschläge sind zu diesem Zeitpunkt zwar zu spät, hätte Walter aber versucht zu innovieren, hätte dies die jetzige Situation verhindern können.

\paragraph*{Sollte AC-Games sein Spieleportal weiterbetreiben trotz finanzieller Schwierigkeiten? \cite{kees_faites_2017}}

Prinzipiell ist gegen dieses Vorhaben nicht viel einzuwenden.
Aus utilitaristischer Sicht ist größerer Nutzen sowohl für AC-Games als auch für die Kunden durch den Weiterbetrieb möglich, sollte dieser ein Erfolg sein.
Sollte der Weiterbetrieb scheitern, so ist dies schlecht für sowohl AC-Games als auch die Kunden.

Wie bereits bei der vorherigen Frage angedeutet, könnte Crowdfunding als Lösung einige Probleme mit sich bringen.
Die finanziellen Schwierigkeiten stammen nicht nur von der Finanzkrise, sondern auch durch Veränderungen am Markt (App-Stores etc.).
Dies bedeutet, dass das Spieleportal für viele ehemalige Kunden keinen Mehrwehrt mehr bietet.
Damit könnte das Crowdfunding des Projekts schwierig werden. 

Man könnte die Frage stellen ob es ethisch ist ein Produkt weiterbetreiben zu wollen, für das viele Kunden mangelnden Nutzen sehen, doch meines Ermessens ist das nur ein schwacher Einwand. 
Auch utilitaristisch ist dies unproblematisch. Kunden sind nicht gezwungen das Produkt zu benutzen.
Kunden, die tatsächlich Mehrwehrt in dem Produkt sehen, können es verwenden und sowohl die Kunden als auch AC-Games profitieren.
Falls das Produkt nicht erfolgreich ist, hat im schlimmsten Fall AC-Games selbst Verluste.

\paragraph*{Welche moralische Verantwortung hat Walter, seine Kunden vor Data Brokers Übernahme zu warnen? \cite{kees_faites_2017}}

Wir haben gesehen, dass Data Broker beabsichtigt, die Übernahme möglichst vor den Nutzern geheimzuhalten.
Auch wird Data Broker die Nutzer negativ in ihrem Kaufverhalten beeinflussen.
Würde Walter die Kunden \emph{nicht} informieren, so werden sie diesem Einfluss unwissend ausgesetzt.
Anders würden die Kunden wissen, dass Data Broker die Firma übernimmt und könnten beispielsweise ihr Konto löschen.
Dies hätte für die Kunden einen höheren Nutzen.

Auf der anderen Seite steht der Profit von AC-Games und damit Data Broker, aber auch der Lebensunterhalt der Mitarbeiter von AC-Games.
Natürlich kann es sein, dass durch Walters Warnung tatsächlich viele Kunden das Spieleportal verlassen und AC-Games noch weniger Einnahmen hat.
Dies könnte dazu führen, dass AC-Games bankrott geht, trotz Übernahme durch Data Broker.
Damit würden auch die Mitarbeiter ihren Arbeitsplatz verlieren. 

Andererseits kann es schwierig sein, die Übernahme dauerhaft geheim zu halten.
Wie in \cref{sec:02:lawethics} besprochen, sind \emph{Soft Costs} auch wichtig zu betrachten, wenn die Konsequenzen einer solchen Entscheidung zu evaluieren sind.
Zwar können auch durch Walters Warnung, wie gerade besprochen, Soft Costs entstehen.
Wenn allerdings versucht wird, die Übernahme geheim zu halten, diese aber später bekannt, dann entstehen noch ernstere Soft Costs, da AC-Games nun zusätzlich mit Recht als hinterlistig bezeichnet werden kann.

Es könnte noch eingewandt werden, dass Walter nicht über die tatsächlich gespeicherten Daten Bescheid wusste.
Allerdings hat Walter als Geschäftsführer die Verantwortung, solch ein wichtiges Vorhaben gründlich zu durchdenken und sich bei seinen Mitarbeitern diesbezüglich hinreichend zu informieren.
Dagegen hätte Walter durch den Datenschutzaudit über die gesammelte Datenmenge informiert sein müssen.

Insgesamt wäre es meines Ermessens utilitaristisch das Richtige, die Kunden zu informieren.
Die Übernahme geheimzuhalten garantiert nicht, dass die Übernahme auch tatsächlich geheim bleibt.
Falls dies später ans Licht kommt, so resultiert dies in ähnlichem Nutzen für die Kunden als wenn Walter sie gleich informiert hätte.
Für AC-Games und Data Broker sind die Nutzen allerdings schlechter.

Hier könnte potentiell Prinzip 13 der ISPP angewandt werden. Dieses Prinzip bezieht sich darauf, dass bei einem \emph{Datenleck} der Nutzer informiert werden soll. Allerdings ist die Situation hier ähnlich genug um derselben ethischen Überlegung zu entsprechen.
Auch hier gelangen die persönlichen Daten des Nutzers unvorhergesehen in die Hände einer anderen Instanz, die auch dubiose Ziele mit diesen Daten verfolgt.

Um den ISPP zu entsprechen sollte Walter ebenfalls die Kunden informieren.

\paragraph*{Welche Verantwortung hatte Kathleen, die Datensammlung zu beenden, nachdem die Daten so lang nicht benutzt wurden? \cite{kees_faites_2017}}

Hier kann Ähnliches eingewandt werden, wie bei der letzten Frage.
Wie bereits zuvor besprochen, ist es sowohl utilitaristisch, als auch nach den ISPP das Richtige, die Kunden über den bevorstehenden Wechsel zu informieren.
Also hätte sie die moralische Verantwortung dazu beizutragen, dass die Nutzer angemessen informiert werden können, vor allem wenn es um Dinge geht die allein in ihrer Macht stehen.

Kathleen kann mit ihrem Wissen über die Datensammlung dazu beitragen.
Tatsächlich scheint sie die Einzige zu sein, die diese Informationen über die gespeicherten Daten an Walter weitergeben kann.
Selbst wenn sie nicht ausreichende Ressourcen hatte, um die Sammlung der Daten zu unterbinden, so hätte es mindestens möglich sein müssen, Walter zu informieren.
Er könnte dann zum Beispiel eine andere Person damit beauftragen, die Systeme umzuprogrammieren.

Dies würde keine Nachteile für Kathleen oder irgendeine andere involvierte Person bedeuten. 
Daher gibt es utilitaristisch keinen Grund, Walter nicht zu informieren.

\paragraph{Ist es problematisch, dass sich AC-Games die Sammlung der Daten vorbehält, obwohl diese nicht benötigt werden?}

Diese Frage lässt sich sowohl utilitaristisch, durch die ISPP und die DSGVO damit beantworten, dass dies problematisch ist.

Diese Daten zu speichern, führt zu keinem Nutzen sowohl für AC-Games als auch für die Kunden.
Das, was das Speichern der Daten bewirkt, ist dass die Daten beispielsweise durch ein Datenleck veröffentlicht oder in die Hände Anderer gelangen könnten.
Damit können also Nachteile entstehen.
Es gibt also utilitaristisch keine Begründung für die Speicherung der Daten.

Auch anhand der ISPP lässt sich dies begründen. Prinzip 6 (nur nötige Daten sammeln) und Prinzip 10 (Daten nur so lange speichern wie nötig) widersprechen der Speicherung der Daten eindeutig.
Darüber hinaus fordert Prinzip 3 (Verständliche Datenschutzrichtlinie und Verwendungszweck), dass der Verwendungszweck der Daten einfach verständlich für den Nutzer aufzufinden ist.
Da die Daten aber keinen Verwendungszweck haben, kann dieser auch nicht auffindbar sein.  

Die DSGVO regelt in Artikel 5 Nr. 1a, dass Art und Umfang der Datenverarbeitung nachvollziehbar und transparent für den Kunden geschildert werden muss.
Würde solch eine Schilderung existieren, so hätte auch Walter sehr einfach herausfinden können, welche Daten tatsächlich gespeichert werden.
Angesichts dessen, dass Walter dies nicht wusste, ist es wahrscheinlich, dass solch eine Schilderung nicht existiert.
Darüber hinaus fordern Artikel 5 Nr. 1b \& 1c, sowie Artikel 6 Nr. 1a, dass eindeutige Verwendungszwecke für die Sammlung und Verarbeitung der Daten festgelegt werden müssen und der Kunde zu diesen zustimmen muss.
Bereits erhobene Daten müssen gelöscht werden, sollten diese nicht (mehr) gebraucht werden (Art. 5 Nr. 1e DSGVO).
Also ist die Speicherung der Daten auch anhand der DSGVO sehr problematisch.

Durch die Nichteinhaltung der DSGVO bricht AC-Games auch Prinzip 1 der ISPP.

\paragraph*{Ist es problematisch, dass der Datenschutzaudit für Data Broker einsehbar war? \cite{kees_faites_2017}}

Ein positiver Nutzen davon, dass der Audit für Data Broker einsehbar war, ist dass dies die öffentliche Zugänglichkeit des Audits impliziert.
Dies erlaubt es den Kunden, oder auch potentiellen Kunden, zu erfahren, welche Daten von AC-Games erhoben und gespeichert werden. Dadurch können die Nutzer informiertere Entscheidungen treffen.

Dagegen könnte gehalten werden, dass durch dieses Wissen Data Broker auf AC-Games erst aufmerksam wurde und dadurch Nachteile für die Nutzer entstehen.
Ich finde diesen Einwand eher schwach. Der Fakt, dass Data Broker Kapital aus dem Datenschutzaudit schlagen konnte ist ein Ergebnis vieler verschiedener Geschehnisse, die zu diesem Punkt geführt haben.

Damit Data Broker aus solch einem Datenschutzaudit profitieren konnte, musste abgesehen von dem Datenschutzaudit folgendes gegeben sein:
\begin{itemize}
    \item AC-Games hat finanzielle Schwierigkeiten, sonst hätte AC-Games nicht an Data Broker verkaufen müssen. 
    \item Data Broker hat genug Mittel, mithilfe von solchen Daten Profit zu erlangen.
    \item Data Broker hat genug Geld, um AC-Games zu kaufen.
\end{itemize}
Die drei oben genanngen Punkte sind spezifische Eigenschaften von AC-Games und Data Broker, die die jetzige Situation herbeigeführt haben. 
Der Datenschutzaudit ist hier eher ein kleinerer Teil der Gesamtsituation.

Daher finde ich die öffentliche Zugänglichkeit des Datenschutzaudits für utilitaristisch gerechtfertigt. Aus Sicht der ISPP und DSGVO sehe ich hier auch keine Probleme.

\paragraph*{War Hanks Lösung sinnvoll? \cite{kees_faites_2017}}

Hanks Verhalten war sowohl illegal als auch unethisch.
Er hat unrechtmäßig gesammelte Daten auf unrechtmäßige Art und Weise genutzt.
Dies ist laut Art. 5 Nr. 1b DSGVO illegal.

Auf ethischer Ebene ist dies auch nach ISPP nicht vertretbar.
Er bricht Prinzip 2 durch seinen unbefugten Zugriff, indem er die Datenbank entgegen jeglicher Vorschriften bearbeitet hat. Ebenso bricht er dieses Prinzip, indem er die Zuordnungstabelle auf seinem USB-Stick speichert.
Dies bringt die darauf gespeicherten Daten in Gefahr gestohlen zu werden oder verloren zu gehen.
Außerdem bricht sein Verhalten eindeutig Prinzip 7.
Er hat das Kaufverhalten der Nutzer zu ihren Ungunsten beeinflusst, ohne wirkliche Begründung.

Aus utilitaristischer Sicht ist Hanks Verhalten auch nicht gutzuheißen.
Die Bearbeitung der Datenbank kann dazu führen, dass die Vereinbarung mit Data Broker nicht mehr gilt. Damit würde AC-Games bankrott gehen und die Mitarbeiter von AC-Games arbeitslos sein.
Wie bereits gesagt, die Übernahme durch Data Broker selbst ist auch ethisch fragwürdig, allerdings scheint Hank nicht über die größeren Konsequenzen dieser Bearbeitung nachgedacht zu haben.

Wenn man über Hanks Motivationen äußerst gütig spekuliert, könnte Hank die Daten aus genau diesem Grund gespeichert haben.
Falls die Vereinbarung mit Data Broker platzen sollte, könnte Hank die Daten wiederherstellen und schlimmeres verhindern.
Allerdings wird sich Data Broker gegen soetwas abgesichert haben.
Data Broker weiß durch den Datenschutzaudit, welche Daten AC-Games gesammelt hat.
Sollte Data Broker bemerken, dass Daten fehlen, so werden sie entweder die fehlenden Daten einfordern, oder AC-Games überhaupt nicht kaufen.
Wenn Hank nun die Daten wiederherstellt, so hat er nichts gewonnen.
Eher könnte Hank nun von Data Broker belangt werden.

Auf der anderen Seite, könnte Hank die Daten auch auf seinem USB-Stick gespeichert haben, um die Daten selbst für seine Zwecke zu benutzen.
Dass dies unethisch wäre, benötigt keine Argumentation. Es wäre einfach Diebstahl.
Außerdem würde Hank dann genau das tun, was er von Data Broker vermutet.
Die Nutzer hätten dabei nicht gewonnen.
Aber auch in diesem Fall würde Data Broker die fehlenden Daten einfordern.

Also haben wir eine \enquote{Lösung}, die nicht zielführend ist, und keiner beteiligten Person hilft. Hank bringt Kundendaten in Gefahr und nutzt diese möglicherweise zu seinen eigenen Gunsten, zum Leidwesen der Kunden.
Die Nutzen dieser \enquote{Lösung} sind also ausschließlich negativ. Also ist Hanks Verhalten auch aus utilitaristischer Sicht zu verurteilen.



