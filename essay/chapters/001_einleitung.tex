\section{Einleitung}

Das Thema Datenschutz wird in der heutigen Zeit von Tag zu Tag relevanter.
Beispielsweise bei Verträgen, durch Bildaufnahmen an öffentlichen Orten, sowie in vielen anderen 
Situationen, werden Daten über Personen erhoben.
Doch ist in den vergangenen Jahrzehnten ein besonders wichtiger Anwendungsbereich entstanden, den Datenschutz betrifft.
Dieser Anwendungsbereich ist das Internet.

Es ist nicht nur so, dass das Internet zu diesem Zeitpunkt von fast fünf Milliarden Menschen genutzt wird. \cite{kemp_digital_2022}
Denn es werden unbeschreibliche Datenmengen von diesen Nutzerinnen und Nutzern sowohl generiert, als auch gesammelt. 
Aus einer Studie von Seagate aus dem Jahr 2018 geht hervor, dass zu diesem Zeitpunkt etwa 33 \emph{Zettabyte} $(10^{21} Bytes)$ an Daten im Umlauf waren. \cite{reinsel_digitization_2018}
Das bedeutet, dass diese Zahl, heute in 2022, höchstwahrscheinlich noch sehr viel größer sein wird.

Die Sammlung dieser Daten kann dabei auf verschiedene Art und Weise erfolgen.
Einerseits können Daten explizit und für die Nutzerinnen und Nutzern deutlich erfolgen. 
Dies ist etwa der Fall, wenn ein Konto für einen Online-Shop erstellt wird, bei dem Namen, Adressen und andere Daten an Unternehmen oder Privatpersonen übergeben werden. 

Darüber hinaus sind auch soziale Medien ein signifikanter Teil der Diskussion um Datenschutz.
Mit schätzungsweise über viereinhalb Milliarden Gesamtnutzern und Nutzerinnen \cite{kemp_digital_2022} ist dies ein Thema, das über $90\%$ der Internetnutzenden betrifft.  
Hier werden oft Nachrichten verschickt und eigene Inhalte, darunter auch persönliche und private Inhalte, in sehr großem Ausmaß hochgeladen. Allein auf YouTube\footnote{\url{https://www.youtube.com/}} wurden zum Anfang des Jahres 2020 \emph{pro Minute} 500 Stunden Videomaterial hochgeladen. \cite{wojcicki_youtube_2020}

Ebenfalls werden häufig Daten ohne das direkte Mitwissen der Nutzerinnen und Nutzer erhoben.
Dies kann etwa durch Tracker auf Internetseiten geschehen, die das Verhalten der Nutzenden während des Besuchs aufzeichnen. 
Das kann prinzipiell bei dem Besuch jeder Website geschehen. 
Selbst wenn also möglicherweise ein allgemeines Verständnis bei den Nutzerinnen und Nutzer davon herrscht, dass durch den Besuch der Website Daten aufgezeichnet werden, 
ist es doch schwierig, möglicherweise sogar unmöglich, konkret zu wissen, welche Daten nun tatsächlich aufgezeichnet wurden.

Wie bereits angedeutet, können diese Daten verschiedenster Art sein.
Insbesondere der Schutz persönlicher Daten wird für die Nutzerinnen und Nutzer von größter Wichtigkeit sein.
Da diese ebenfalls, wie zuvor beschrieben, entweder ausdrücklich oder im Hintergrund gesammelt werden,
ist es wichtig, dass diese Daten auch angemessen geschützt werden.

Dazu müssen wir uns erst mit den Aufgaben befassen, die Datenschutz betreffen, sowie den Folgen, sollte Datenschutz nicht eingehalten worden sein.
Wir betrachten hierzu das Paper von Lee et al.\ \enquote{An Ethical Approach to Data Privacy Protection}. \cite{lee_ethical_2016}


