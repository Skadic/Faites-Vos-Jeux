\begin{frame}{International Data Privacy Principles}

    % Einleitung
    \note<1>[item]{13 Ethische Prinzipien für Umgang mit Daten für Firmen}
    \note<1>[item]{Erarbeitet von Wolfang Zankl - Institut für Zivilrecht Wien}
    \note<1>[item]{Orientieren sich an den Datenschutzstandards von Ländern weltweit}
    \note<1>[item]{Viele davon relativ selbsterklärend.}
    \note<1>[item]{Meine Übersetzung}


    % Prinzip anzeigen
    \visible<2->{Eine Firma, die persönliche Daten nutzt, soll\dots}

    \only<2->{\begin{block}{Prinzip 1 \cite{zankl_international_2014}}
        \dots die jeweiligen nationalen and internationalen Regelungen einhalten die Datenschutz als Gegenstand haben. 
    \end{block}}

    % Genauer Verwendungszweck
    \note<3>[item]{Art. 5 Abs 1b: Genauer Verwendungszweck \begin{itemize}
        \item Kennen zwar ihre AGB nicht, aber
        \item Für die meisten Daten $\exists$ kein Verwendungszweck
        \item $\Rightarrow$ Es kann auch keiner angegeben worden sein.
    \end{itemize}}

    \only<3->{
        \textcolor{red}{Nein} 
    }
    \only<3>{Art. 5 Abs. 1b DSGVO: Genauer Verwendungszweck muss exakt und eindeutig festgehalten werden. }
    
    % Sammlung von Daten außerhalb von designiertem Zweck nicht erlaubt 
    \note<4>[item]{Nicht nur Walter etc. verstoßen dagegen sondern auch Hank \begin{itemize}
        \item Sein Experiment: Unauthorisierte Nutzung
    \end{itemize}}

    \only<4>{Art. 5 Abs. 1c DSGVO: Jegliche Sammlung und Nutzung außerhalb dessen ist nicht zulässig.}
    
    % Löschung erhobener Daten
    \note<5>[item]{Wer immer Verantwortung für Datenspeicherung hat (Walter, Kathleen, etc.), hat es nicht getan}

    \only<5>{Art. 5 Abs. 1e DSGVO: Erhobene Daten müssen gelöscht werden, sollten sie nicht mehr gebraucht werden.}

\end{frame}

\begin{frame}{International Data Privacy Principles}

    \note<1->[item]{
        Hank hat definitiv gebrochen. \begin{itemize}
            \item Laden der Zuordnungstabelle auf einen privaten USB Stick: 
        \end{itemize}
    }

    Eine Firma, die persönliche Daten nutzt, soll\dots

    \begin{block}{Prinzip 2 \cite{zankl_international_2014}}
        \dots durch Einhalten aktueller Sicherheitsstandards, unbefugten Zugriff und Verarbeitung, sowie Löschung und Verlust zu verhindern.
    \end{block}

    \only<2>{
        AC-Games: \textcolor{gold}{Vielleicht} Wissen nicht viel über die \emph{Sicherung} der Daten.
    }
    \only<3->{
        Hank: \textcolor{red}{Nein} Er hat die Zuordnungstabelle auf einen USB Stick geladen. Das kann einfach zu Verlust führen.
        }

\end{frame}

\begin{frame}{International Data Privacy Principles}

    \note<1->[item]{
        Datenschutzrichtlinie, die Zweck zeigt, kann nicht existieren \begin{itemize}
          \item es gab keinen Zweck
        \end{itemize}
    }

    Eine Firma, die persönliche Daten nutzt, soll\dots

    \begin{block}{Prinzip 3 \cite{zankl_international_2014}}
        \dots eine für Kunden einfach verständliche Datenschutzrichtlinie aufstellen und den Kunden erlauben die verantwortliche Person persönlich zu kontaktieren.
        Darüber hinaus muss die Richtline beschreiben welche Daten zu welchem Zweck erhoben werden [und] wie die Daten genutzt werden [\dots].
    \end{block}

    \only<2->{
        \textcolor{red}{Nein} Bereits gesagt: Es gab keinen Zweck für viele der Daten, deswegen kann auch niemand etwas über den Zweck sagen.
    }

\end{frame}

\begin{frame}{International Data Privacy Principles}

    \note<1->[item]{Könnte man sagen, aber trotzdem dubios}
    \note<2->[item]{Frage: Darf man sowas mit AGB überhaupt machen?}

    Eine Firma, die persönliche Daten nutzt, soll\dots

    \begin{block}{Prinzip 5 \cite{zankl_international_2014}}
        \dots Daten nicht an unauthorisierte Dritte weiter geben, außer [\dots] der Kunde hat zugestimmt. [\dots] 
    \end{block}

    \only<2->{
        \textcolor{green}{Ja?} Scheinbar müssen bei der Übernahme die AGB nicht geändert werden. Also müsste die Übernahme unter den vorherigen AGB rechtens sein. 
    }

\end{frame}

\begin{frame}{International Data Privacy Principles}

    \note<1->[item]{Einfache Sache}

    Eine Firma, die persönliche Daten nutzt, soll\dots

    \begin{block}{Prinzip 6 \cite{zankl_international_2014}}
        \dots nur nötige Daten sammeln.
    \end{block}

    \only<2->{
        \textcolor{red}{Nein} Offensichtlich
    }

\end{frame}

\begin{frame}{International Data Privacy Principles}

    Eine Firma, die persönliche Daten nutzt, soll\dots

    \begin{block}{Prinzip 7 \cite{zankl_international_2014}}
        \dots die gesammelten Daten auf faire Art und Weise benutzen [\dots] und nur zum Zweck der für die Firma nötigen Zwecke.
    \end{block}

    \only<2>{
        AC-Games: \textcolor{green}{Ja} Sie haben zwar zu viele Daten gesammelt, aber wirklich nur Daten benutzt, die sie brauchen.
    }
    \only<3->{
        Hank: \textcolor{red}{Nein} Er hat unauthorisiert gesammelte Daten zu unauthorisierten Zwecken benutzt um die Nutzer zu ihren Ungunsten zu beeinflussen.
    }

\end{frame}

\begin{frame}{International Data Privacy Principles}

    Eine Firma, die persönliche Daten nutzt, soll\dots

    \begin{block}{Prinzip 8 \cite{zankl_international_2014}}
        \dots keine Daten an Instanzen weitergeben, die internationale Datenschutzstandards nicht einhalten. 
    \end{block}

    \only<2->{
        \textcolor{orange}{Dubios} Wir wissen nicht \emph{genau}, dass Data Broker schlecht mit den Daten umgeht, aber es ist doch sehr naheliegend.
    }

\end{frame}

\begin{frame}{International Data Privacy Principles}

    Eine Firma, die persönliche Daten nutzt, soll\dots

    \begin{block}{Prinzip 10 \cite{zankl_international_2014}}
        \dots Daten nur so lang wie nötig aufbewahren.
    \end{block}

    \only<2->{
        \textcolor{red}{Nein} Die Daten waren nie nötig, also ist der erlaubte Speicherzeitraum genau null Sekunden. 
    }

\end{frame}
